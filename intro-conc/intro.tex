\section{La concurrencia está aquí}

\subsection{The free lunch is over}


\begin{frame}[t]{Evolución}
\begin{itemize}
  \item Hasta 2005 (aprox.) incremento sostenido en frecuencias de reloj.
    \begin{itemize}
      \item Incrementos del rendmiento en torno al 52\% anual.
    \end{itemize}
  \vspace{12pt}
  \pause
  \item Pero entonces \ldots
    \begin{itemize}
      \item Se acabó la fiesta (\emph{The free lunch is over}).
      \item Estabilización de frecuencias de reloj en torno a 3GHz.
      \item La industria apuesta por multi/many-core.
    \end{itemize}
  \vspace{12pt}
  \pause
  \item La concurrencia y el paralelismo se convierten en más relevantes.
    \begin{itemize}
      \item \alert{Recuerda}: No son lo mismo pero muy interrelacionados.
    \end{itemize}
\end{itemize}
\end{frame}

\begin{frame}[t]{Concurrencia y paralelismo}
\begin{itemize}
  \item \textbf{Concurrencia}: Ejecución de varias tareas de forma que el inicio de una
se encuentra entre el principio y final de otra.
    \begin{itemize}
      \item Puede darse en un único procesador.
      \item Objetivos:
        \begin{itemize}
          \item Mejorar aprovechamiento de recursos (trabajos por unidad de tiempo).
          \item Mejorar experiencia de respuesta en presencia de E/S.
          \item Mejorar estructura de programas basados en tareas.
        \end{itemize}
    \end{itemize}

  \vspace{12pt}
  \pause
  \item \textbf{Paralelismo}: Ejecución de varias tareas de forma simultánea.
    \begin{itemize}
      \item Típicamente cada tarea usa un procesador (o un core).
      \item Objetivos:
        \begin{itemize}
          \item Mejorar el rendimiento (tiempo de un trabajo).
          \item Mejorar la escalabilidad.
        \end{itemize}
    \end{itemize}
\end{itemize}
\end{frame}

\subsection{C++ y la concurrencia}

\begin{frame}[t]{¿Por qué?}
\begin{itemize}
  \item Portabilidad:
    \begin{itemize}
      \pause
      \item Antes de 2011 diversos modelos de programación concurrente.
        \begin{itemize}
          \item Windows Threads, POSIX Threads, \ldots
          \item Extensiones en el compilador: \cppid{\_\_declspec( thread )},
\cppid{\_\_thread}, \ldots
        \end{itemize}
      \pause
      \item Necesidad de código no portable para operaciones libres de cerrojos.
        \begin{itemize}
          \item Típicamente en ensamblador.
        \end{itemize}
    \end{itemize}
  \pause
  \item Completitud y corrección:
    \begin{itemize}
      \item Un modelo de hilos no puede implementarse sin soporte del
compilador.
        \begin{itemize}
          \item H. Bohem. \emph{Threads cannot be implemented as a library}.
PLDI'2005.
        \end{itemize}
    \end{itemize}
\end{itemize}
\end{frame}

\begin{frame}[t]{Y además \ldots}
\begin{itemize}
  \item Formalización del modelo de memoria.
    \begin{itemize}
      \item Hasta 2005 varios fabricantes no habían clarificado su modelo
            de consistencia de memoria.
      \item Formalización compleja.
      \item Problemas de implementación en lenguajes (Java, C, C++).
      \item Efecto: Clarificación de los fabricantes de procesadores.
    \end{itemize}
  \pause
  \vspace{12pt}
  \item Efecto sobre C11.
    \begin{itemize}
      \item Incluisión de un modelo de hilos muy similar.
      \item Portabilidad de hilos en C (mejor que pthreads).
    \end{itemize}
\end{itemize}
\end{frame}

\begin{frame}[t,fragile]{¿Qué no está incluido?}
\begin{itemize}
  \item Paralelismo: ISO/IEC TS 19570.
\begin{lstlisting}
sort(par, v.begin(), v.end());
transform(vec, v.begin(), v.end(), [](auto x) { return x*2; });
\end{lstlisting}
  \item Más concurrencia: ISO/IEC TS 19571.
\begin{lstlisting}
thread_pool pool;
for (int i=0;i<max;++i) v.push_back(add(pool, calcula, i));
for (auto & x : v) res += x.get();
\end{lstlisting}
  \item Memoria transaccional.
\begin{lstlisting}
void cuenta::ingresa(double x) {
  atomic_noexcept {
    saldo += x;
  }
}
\end{lstlisting}
\end{itemize}
\end{frame}

\begin{frame}[t]{Elementos}
\begin{itemize}
  \item Un nuevo modelo de memoria.
  \pause
  \item Almacenamiento local al hilo estándar (\cppkey{thread\_local}).
  \pause
  \item Una interfaz simple de alto nivel para ejecución de tareas.
    \begin{itemize}
      \item \cppid{async} y \cppid{future}.
    \end{itemize}
  \pause
  \item Una interfaz más detallada para ejecución de hilos.
    \begin{itemize}
      \item \cppid{thread} y \cppid{promise}
    \end{itemize}
  \pause
  \item Mecanismos de sincronización.
    \begin{itemize}
      \item \emph{Mutex}, cerrojos y variables condición.
    \end{itemize}
  \pause
  \item Una interfaz portable de bajo nivel para programación \emph{lock-free} y
\emph{wait-free}.
    \begin{itemize}
      \item Tipos \cppid{atomic<>}.
    \end{itemize}
\end{itemize}
\end{frame}
