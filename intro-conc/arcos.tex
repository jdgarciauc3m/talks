\section{ARCOS@uc3m}

\begin{frame}{La Universidad Carlos III de Madrid}
\begin{itemize}
  \item Una universidad joven.
    \begin{itemize} 
      \item Fundada en 1989.
      \item Más del 80\% del profesorado es menor de 50 años.
    \end{itemize}
  \item \pause Orientación a la investigación:
    \begin{itemize}
      \item 1.100 profesores con grado de doctor.
      \item Mayor proporción de profesores con investigación evaluada positivamente en España.
    \end{itemize}
  \item \pause Clara vocación internacional:
    \begin{itemize}
      \item Triplicada la financiación de proyectoes europeos en el periodo 2007-2012.
      \item Una de las 10 universidade españolas con más éxito en proyectos europeos en ICT.
      \item Una de las 50 mejores universidades de menos de 50 años.
      \item Campus de excelencia internacional.
      \item Titulaciones de ingeniería acreditadas EUR-ACE.
    \end{itemize}
\end{itemize}
\end{frame}

\begin{frame}{El grupo ARCOS}
  \begin{itemize}
    \item Grupo de Arquitectura de Computadores, Comunicaciones y Sistemas.
      \begin{itemize}
        \item Fundado en 1999.
        \item 20 investigadores.
      \end{itemize}
    \item \pause Orientación.
      \begin{itemize}
        \item Investigación aplicada orientada a la generación de valor.
        \item Actividad en principales redes de investigación nacionales y eruopeas.
        \item Cooperación con principales centros de investigación en Europa y USA.
      \end{itemize}
    \item \pause Líneas de investigación:
      \begin{itemize}
        \item Computación y E/S de altas prestaciones.
        \item \emph{Big data}: Captación, distribución y análisis de datos.
        \item Sistemas ciberfísicos.
        \item Modelos de programación para la mejora de aplicaciones.
      \end{itemize}
  \end{itemize}
\end{frame}

\begin{frame}
  \begin{itemize}
    \item Computación y E/S de altas prestaciones
      \begin{itemize}
        \item Sistemas de almacenamiento de altas prestaciones: \emph{Expand}, \emph{AHPIOS}, \emph{HiperVoCo}.
        \item Optimizaciones de MPI: \emph{CoMPI}, \emph{Dynamic CoMPI}.
        \item Simulación de Infraestructura: \emph{iCanCloud}, \emph{SIMCAN}
        \item Simulación de sistemas de ingeniería: \emph{SIRTE}, \emph{SYCE}, \emph{RAIL-RCM}.
        \item Aplicaciones biomédicas: \emph{EpiGraph}, imagen médica.
        \item Sistema \emph{ulta-scale}: Acción COST \textbf{NESUS}.
      \end{itemize}
    \vspace{1em}
    \item \pause \emph{Big data}: Captación, distribución y análisis de datos.
      \begin{itemize}
        \item Predicción de carga y adaptación de de infraestructura: \emph{ADAPTIT}, análisis de redes sociales.
        \item Distribución masiva de datos y p2p: \emph{HIDDRA}.
        \item Infraestructras Cloud y elasticidad: \emph{CONDESA}.
      \end{itemize}
  \end{itemize}
\end{frame}

\begin{frame}
  \begin{itemize}
    \item Sistemas ciberfísicos.
      \begin{itemize}
        \item Middlewares para redes de sensores: \emph{OSAL}.
        \item Monitorización de grandes infraestructuras: \emph{SCGMM}.
        \item Sistemas de aviónica: \emph{SEAS3}.
        \item Desarrollo basado en modelos para sistemas aeroespaciales: \emph{PERIGEO}.
      \end{itemize}
    \vspace{1em}
    \item \pause Modelos de programación para la mejora de aplicaciones.
      \begin{itemize}
        \item Mejora de prestaciones: \emph{REPARA} (FP7).
        \item Normalización de C++ (ISO 14882).
        \item Refactorización de aplicaciones: multi/many-core, GPGPU, aceleradores.
        \item Mantenibilidad de aplicaciones y vulnerabilidades debidas al lenguaje.
      \end{itemize}
  \end{itemize}
\end{frame}

\begin{frame}{Oferta tecnológica}
  \begin{itemize}
    \item Optimización de aplicaciones.
      \begin{itemize}
        \item Paralelización de aplicaciones.
        \item Multi-core y mult-GPU.
        \item Migración de aplicaciones a \emph{cloud}.
      \end{itemize}
    \item Modelado inteligente y simulación de sistemas.
      \begin{itemize}
        \item Modelado de infraestructura ferroviaria.
        \item Simulación de sistemas aeroespaciales.
        \item Simulación de sistemas TI (\emph{cluster} y \emph{cloud}).
      \end{itemize}
    \item Diseño y programación de sistemas empotrados y de tiempo real
      \begin{itemize}
        \item Sistemas de aviónica.
        \item Planificación multimedia.
        \item Diseño basado en modelos.
        \item Técnicas de ahorro de energía en \emph{data centers}.
      \end{itemize}
  \end{itemize}
\end{frame}

\begin{frame}{Oferta formativa}
  \begin{itemize}
    \item Desarrollo en C++.
      \begin{itemize}
        \item C++11/14.
        \item Concurrencia portable en C++11.
        \item Programación paralela en C++.
      \end{itemize}
    \vspace{1em}
    \item Desarrollo de software de altas prestaciones.
      \begin{itemize}
        \item CUDA, OpenMP, MPI, TBB, \ldots
      \end{itemize}
    \vspace{1em}
    \item Administración de sistemas.
      \begin{itemize}
        \item Sistemas clúster.
        \item Sistemas \emph{cloud}.
      \end{itemize}
  \end{itemize}
\end{frame}
