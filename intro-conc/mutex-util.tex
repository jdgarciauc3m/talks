\subsection{Utilidades para mutex}

\begin{frame}[fragile]{Adquisición y liberación como RAII}
\begin{itemize}
  \item Cada adquisición de un cerrojo debe encajar con una liberación del mismo cerrojo.
  \item Problemas:
    \begin{itemize}
      \item Código con excepciones.
      \item Funciones con múltiples puntos de retorno.
      \item Orden de adquisición y liberación de cerrojos.
    \end{itemize}
\begin{lstlisting}
void f(mutex & m1, mutex & m2, vector<registro> & v, vector<registro> & w) {
  m1.lock();
  m2.lock();
  
  if (v.size() == 0) return;
  procesa(v,w); // Podría lanzar excepciones.
 
  m1.unlock();
  m2.unlock();
}
\end{lstlisting}
\end{itemize}
\end{frame}

\begin{frame}[fragile]{\texttt{lock\_guard}}
\begin{itemize}
  \item \cppid{lock\_guard<M> l\{m\}}: Construye una guarda que adquiere el \emph{mutex} \cppid{m}.
  \item \cppid{lock\_guard<M> l\{m, adopt\_lock\}}: Construye una guarda que asume que el \emph{mutex} \cppid{m} ya
        se encuentra adquirido por el hilo actual.
  \item \emph{Destructor}: Libera el \emph{mutex} asociado a la guarda.
\begin{lstlisting}
void f(mutex & m1, mutex & m2, vector<registro> & v, vector<registro> & w) {
  lock_guard<mutex> l1{m1};
  lock_guard<mutex> l2{m2};
  
  if (v.size() == 0) return;
  procesa(v,w); // Podría lanzar excepciones.
} // m1 y m2 se liberan en el orden correcto
\end{lstlisting}
\end{itemize}
\end{frame}

\begin{frame}{\texttt{unique\_lock}}
\begin{itemize}
  \item Gestión de adquisición/liberación más general.
  \item Construcción:
    \begin{itemize}
      \item \cppid{unique\_lock<M> l\{\}}: Construye un cerrojo sin \emph{mutex} asociado.
      \item \cppid{unique\_lock<M> l\{m\}}: Construye un cerrojo que adquiere el \emph{mutex} \cppid{m}.
      \item \cppid{unique\_lock<M> l\{m, defer\_lock\}}: Construye un cerrojo asociado al mutex \emph{mutex} \cppid{m}, pero no lo adquiere.
      \item \cppid{unique\_lock<M> l\{m, try\_lock\}}: Construye un cerrojo asociado al mutex \emph{mutex} \cppid{m}, y ejecuta \cppid{m.try\_lock()}.
        \begin{itemize}
          \item Potencialmente puede adquirir \cppid{m}.
        \end{itemize}
      \item \cppid{unique\_lock<M> l\{m, adopt\_lock\}}: Construye un cerrojo asociado al mutex \emph{mutex} \cppid{m}, y asume que ya ha sido adquirido por el hilo actual.
        \begin{itemize}
          \item El destructor realizará una liberación.
        \end{itemize}
    \end{itemize}
\end{itemize}
\end{frame}

\begin{frame}{Más formas de iniciación}
\begin{itemize}
  \item Operaciones de construcción:
    \begin{itemize}
      \item \cppid{unique\_lock<M> l\{m,t\}}: Construye un cerrojo asociado al mutex \emph{mutex} \cppid{m}, y ejecuta \cppid{m.try\_lock\_until(t)}.
      \item \cppid{unique\_lock<M> l\{m,d\}}: Construye un cerrojo asociado al mutex \emph{mutex} \cppid{m}, y ejecuta \cppid{m.try\_lock\_for(d)}.
    \end{itemize}
  \item Movimiento:
    \begin{itemize}
      \item \cppid{unique\_lock<M> l\{l2\}}: Mueve el \emph{mutex} del cerrojo \cppid{l2} al nuevo cerrojo \cppid{l}.
      \item \cppid{l = move(l2)}: Mueve el \emph{mutex} del cerrojo \cppid{l2} al cerrojo \cppid{l}.
    \end{itemize}
  \item Destrucción: 
    \begin{itemize}
      \item Libera el \emph{mutex} asociado al cerrojo.
    \end{itemize}
  \item Intercambio:
    \begin{itemize}
      \item \cppid{l1.swap(l2)}.
      \item \cppid{swap(l1,l2)}.
    \end{itemize}
\end{itemize}
\end{frame}

\begin{frame}{Operaciones}
\begin{itemize}
  \item Operaciones de adquisición/liberación absolutas:
    \begin{itemize}
      \item \cppid{l.lock()} $\rightarrow$ \cppid{m.lock()}.
      \item \cppid{l.try\_lock()} $\rightarrow$ \cppid{m.try\_lock()}.
      \item \cppid{l.try\_unlock()} $\rightarrow$ \cppid{m.try\_unlock()}.
    \end{itemize}
  \item Operaciones de adquisición/liberación temporizadas:
    \begin{itemize}
      \item \cppid{l.try\_lock\_for(d)} $\rightarrow$ \cppid{m.try\_lock\_for(d)}.
      \item \cppid{l.try\_lock\_until(t)} $\rightarrow$ \cppid{m.try\_lock\_until(t)}.
    \end{itemize}
  \item Operaciones de acceso:
    \begin{itemize}
      \item \cppid{pm = l.release()}: \cppid{l} deja de gestionar el \emph{mutex} \cppid{*pm}.
      \item \cppid{pm = l.mutex()}: Obtiene el\emph{mutex} \cppid{*pm} gestionado por \cppid{l}.
    \end{itemize}
\end{itemize}
\end{frame}

\begin{frame}[fragile]{Problema de adquisición múltiple}
\begin{itemize}
  \item Si varios componentes adquieren un conjunto de cerrojos uno a uno en el mismo orden:
    \begin{itemize}
      \item Se puede producir \emph{deadlock}.
    \end{itemize}
\begin{lstlisting}
void f(mutex & m1, mutex & m2, mutex & m3) {
  // Puede haber reordenación
  unique_lock<mutex> l1{m1};
  unique_lock<mutex> l2{m2};
  unique_lock<mutex> l3{m3};
  // ...
}

void g() {
  mutex a, b, c;
  thread t1{f, ref{m1}, ref{m2}, ref{m3}};
  thread t2{f, ref{m2}, ref{m3}, ref{m1}};
  // ...
}
\end{lstlisting}
\end{itemize}
\end{frame}

\begin{frame}[fragile]{Algoritmos de adquisicón múltiple}
\begin{itemize}
  \item Algoritmos con argumentos variables para la adquisición de múltiples cerrojos.
    \begin{itemize}
      \item \cppid{r = try\_lock(l...)}: Adquiere todos los cerrojos en orden. 
        \begin{itemize}
          \item \cppid{r=-1} $\rightarrow$ se tuvo éxito en adquirirlos todos.
          \item \cppid{r=k} $\rightarrow$ \cppid{k} No se pudo adquirir el \emph{k-ésimo} cerrojo.
        \end{itemize}
      \item \cppid{lock(l...)}: Adquiere todos los cerrojos sin posibilidad de \emph{deadlock}.
    \end{itemize}
\begin{lstlisting}
void f(mutex & m1, mutex & m2, mutex & m3) {
  unique_lock<mutex> l1{m1, defer_lock};
  unique_lock<mutex> l2{m2, defer_lock};
  unique_lock<mutex> l3{m3, defer_lock};
  lock(l1,l2,l3);
  // ...
}
\end{lstlisting}
\end{itemize}
\end{frame}

