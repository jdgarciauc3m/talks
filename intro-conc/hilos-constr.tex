\subsection{Construcción e identidad}

\begin{frame}[fragile]{Construcción de hilos}
\begin{itemize}
  \item Un hilo se construye con una función y los argumentos que se debe pasar a la función.
    \begin{itemize}
      \item Plantilla con número variable de argumentos.
      \item Seguro en tipos.
    \end{itemize}
\begin{lstlisting}
void f();
void g(int, double);

thread t1{f}; // OK
thread t2{f, 1}; // Error: demasiados argumentos.

thread t3{g, 1, 0.5}; // OK
thread t4{g}; // Error: faltan argumentos.
thread t5{g, 1, "Hola"}; // Error: tipos incorrectos
\end{lstlisting}
\end{itemize}
\end{frame}

\begin{frame}[fragile]{Construcción y referencias}
\begin{itemize}
  \item El constructor de \cppid{thread} es una plantilla con argumentos variables.
\begin{lstlisting}
template <class F, class ...Args> 
explicit thread(F&& f, Args&&... args);
\end{lstlisting}
  \pause
  \item El paso de parámetros a un hilo es por valor.
  \item Para forzar el paso de parámetros por referencia:
    \begin{itemize}
      \item Usar una función de ayuda para \cppid{reference\_wrapper}.
      \item Usar lambdas y capturas por referencia.
    \end{itemize}
\begin{lstlisting}
void f(registro & r);

void g(registro & s) {
  thread t1{f,s};
  thread t2{f, ref(s)};
  thread t3{[&] { f(s); }};
}
\end{lstlisting}
\end{itemize}
\end{frame}

\begin{frame}[fragile]{Construcción en dos etapas}
\begin{itemize}
  \item La construcción incluye el inicio de la ejecución del hilo.
    \begin{itemize}
      \item No hay operación separada para \emph{iniciar} la ejecución.
    \end{itemize}
\begin{lstlisting}
struct productor {
  productor(cola<peticiones> & c);
  void operator()();
  // ...
};

struct consumidor {
  consumidor(cola<peticiones> & c);
  void operator()();
  // ...
};

cola<peticiones> c;
productor prod{c};
consumidor cons{c};

thread tp{prod};
thread tc{cons};
\end{lstlisting}
\end{itemize}
\end{frame}

