\section{Concurrencia mínima}

\subsection{Tareas asíncronas}

\begin{frame}[t]{Historia de dos tareas}
\begin{block}{tasks.h}
\lstinputlisting{async/tasks.h}
\end{block}
\end{frame}

\begin{frame}[t,fragile]{Ejecutando las tareas secuencialmente}
\begin{block}{main1.cpp}
\lstinputlisting{async/main1.cpp}
\end{block}
\begin{lstlisting}[style=terminal]
$ time ./test1
++++++++++********************
real	0m0.158s
user	0m0.004s
sys	0m0.000s
\end{lstlisting}

\begin{itemize}
  \item Ambas tareas se pueden ejecutar concurrentemente.
\end{itemize}

\end{frame}


\begin{frame}[t,fragile]{Ejecutando las tareas concurrentemente}
\begin{block}{main2.cpp}
\lstinputlisting{async/main2.cpp}
\end{block}
\end{frame}

\begin{frame}[t,fragile]{Concurrencia}
\begin{lstlisting}[style=terminal]
$ time ./test1
+*+**+*+*+*+*+*+*+*+**********
real	0m0.106s
user	0m0.004s
sys	0m0.000s
\end{lstlisting}

\begin{itemize}
  \item Salida entrelazada a \cppid{cout}.
  \item Aprovechamiento del tiempo.
\end{itemize}

\end{frame}

\begin{frame}[t,fragile]{Políticas de ejecución}
\begin{itemize}
  \item La biblioteca ofrece dos políticas de ejecución.
    \begin{itemize}
      \item \cppid{std::launch::async}.
        Ejecución asíncrona \emph{posiblemente} (as-if) en otro hilo.
      \item \cppid{std::launch::deferred}.
        Ejecución en el mismo hilo cuando se invca a \cppid{get}.
    \end{itemize}
  \pause
  \item ¿Y si no se especifica la política?
\begin{lstlisting}[basicstyle=\small\ttfamily]
std::async(f);
\end{lstlisting}
    \begin{itemize}
      \item Como si se invocase a \cppid{std::launch::async} \cppkey{|} \cppid{std::launch::deferred}.
      \item La implementación decide.
    \end{itemize}
\end{itemize}
\end{frame}

\subsection{Parámetros y resultados de una tarea}

\begin{frame}[t]{Una tarea de cálculo}
\begin{block}{main3.cpp (I)}
\lstinputlisting[lastline=19]{async/main3.cpp}
\end{block}
\end{frame}

\begin{frame}[t]{Pasando parámetros y recogiendo resultados}
\begin{block}{main3.cpp (y II)}
\lstinputlisting[firstline=21]{async/main3.cpp}
\end{block}
\end{frame}

\begin{frame}[t]{Una tarea excepcional}
\begin{block}{main4.cpp (I)}
\lstinputlisting[lastline=13]{async/main4.cpp}
\end{block}
\end{frame}

\begin{frame}[t]{Una tarea excepcional}
\begin{block}{main4.cpp (y III)}
\lstinputlisting[firstline=13]{async/main4.cpp}
\end{block}
\end{frame}
