\subsection{Terminación de hilos}

\begin{frame}[fragile]{Unión de tareas}
\begin{itemize}
  \item Cuando un hilo desea esperar a que otro hilo finalice puede usar la operación \cppid{join()}.
    \begin{itemize}
      \item \cppid{t.join()} $\rightarrow$ espera a que t haya finalizado.
    \end{itemize}
\end{itemize}
\begin{block}{hilos/join1.cpp}
\lstinputlisting[lastline=15]{hilos/join1.cpp}
\end{block}
\end{frame}

\begin{frame}[fragile]{Ejemplo}
\begin{block}{progreso.cpp}
\begin{lstlisting}
void actualiza_barra() {
  while (!tarea_terminada()) {
    this_thread::sleep_for(chrono::second(1))
    actualiza_progreso();
  }
}

void f() {
  thread t{actualiza_barra};
  alguna_otra_cosa();
  t.join();
}
\end{lstlisting}
\end{block}
\end{frame}

\begin{frame}[fragile]{¿Qué pasa si se olvida \texttt{join}?}
\begin{itemize}
  \item Si se sale del alcance donde se define el hilo, se invoca al destructor.
  \item \alert{Problema}: Se podría perder el vínculo con un hilo del sistema que se seguiría
        ejecutando y al que no se podría acceder.
    \begin{itemize}
      \item Si no se ha hecho \cppid{join()} el destructor llama a \cppid{terminate()}.
    \end{itemize}
\begin{lstlisting}
void actualiza() {
  for (;;) {
    muestra_reloj(stead_clock::now());
    this_thread::sleep_for(second{1});
  }
}

void f() {
  thread t(actualiza);
}
\end{lstlisting}
  \item Se ejecuta \cppid{terminate()} al abandonar \cppid{f()}.
\end{itemize}
\end{frame}

\begin{frame}[fragile]{Problemas con la destrucción}
\begin{lstlisting}
void f();
void g();

void ejemplo() {
  thread t1{f}; // Hilo que ejecuta la tarea f
  thread t2; // Hilo vacío.

  if (modo == modo1) {
    thread tg {g}; 
    // ...
    t2 = move(tg); // tg vacía, t2 asociada a g()
  }

  vector<int> v{10000}; // Podría lanzar excepciones
  t1.join();
  t2.join();
}
\end{lstlisting}
\begin{itemize}
  \item ¿Qué ocurre si el constructor de \cppid{v} lanza una excepción.
  \item ¿Qué ocurre si se llega al final con \cppid{modo==modo1}?
\end{itemize}
\end{frame}

\begin{frame}[fragile]{Hilo automático}
\begin{block}{Se puede usar el patrón RAII}
\begin{lstlisting}
struct hilo_automatico : thread {
  using thread::thread;
  ~hilo_automatico() { 
    if (t.joinable()) t.join(); 
  }
};
\end{lstlisting}
\end{block}
\begin{block}{Simplificación de código y seguridad}
\begin{lstlisting}
void ejemplo() {
  hilo_automatico t1{f}; // Hilo que ejecuta la tarea f
  hilo_automatico t2; // Hilo vacío.

  if (modo == modo1) {
    hilo_automatico tg {g}; 
    // ...
    t2 = move(tg); // tg vacía, t2 asociada a g()
  }

  vector<int> v{10000}; // Podría lanzar excepciones
}
\end{lstlisting}
\end{block}
\end{frame}

\begin{frame}[fragile]{Hilos no asociados}
\begin{itemize}
  \item Se puede usar \cppid{detach()} para indicar que un hilo siga ejecutando 
        después de que el destructor se ejecute.
  \item Útil para tareas que se ejecutan como demonios.
\begin{lstlisting}
void actualiza() {
  for (;;) {
    muestra_reloj(stead_clock::now());
    this_thread::sleep_for(second{1});
  }
}

void f() {
  thread t(actualiza);
  t.detach();
}
\end{lstlisting}
\end{itemize}
\end{frame}

\begin{frame}[t]{Problemas con hilos no asociados}
\begin{itemize}
  \item \alert{Incovenientes}:
    \begin{itemize}
      \item Se pierde el control de qué hilos están activos.
      \item No se sabe si se puede usar el resultado generado por un hilo.
      \item No se sabe si un hilo ha liberado sus recursos.
      \item Se podría acabar accediendo a objetos que han sido destruidos.
    \end{itemize}
  \pause
  \vspace{1em}
  \item \alert{Recomendaciones}:
    \begin{itemize}
      \item Evitar el uso de hilos no asociados.
      \item Mover los hilos a otro alcance (via valor de retorno).
      \item Mover los hilos a un contenedor en un alcance mayor.
    \end{itemize}
\end{itemize}
\end{frame}


