\subsection{Almacenamiento local al hilo}

\begin{frame}{Variables locales a hilos}
\begin{itemize}
  \item Alternativa a \cppkey{static} como especificador de almacenamiento: \cppkey{thread\_local}.
    \begin{itemize}
      \item Una variable \cppkey{static} tiene una única copia compartida por todos los hilos.
      \item Una variable \cppkey{thread\_local} tiene una copia por cada hilo.
    \end{itemize}
  \pause
  \item Tiempo de vida: duración de almacenamiento de hilo (\emph{thread storage duration}).
    \begin{itemize}
      \item Se inicia antes de su primer uso en el hilo.
      \item Se destruye en la salida del hilo.
    \end{itemize}
  \pause
  \item Razones para usar almacenamiento local al hilo:
    \begin{itemize}
      \item Transformar datos de almacenamiento estático a almacenamiento local al hilo.
      \item Mantener cachés de datos locales a cada hilo (acceso exclusivo).
        \begin{itemize}
          \item Importante en máquinas con caché separada y protocolos de coherencia.
        \end{itemize}
    \end{itemize}
\end{itemize}
\end{frame}

\begin{frame}[fragile]{Ejemplo}
\begin{block}{Una lista de claves}
\begin{lstlisting}
thread_local map<int, int> cache;

int calcula_clave(int x) {
  auto i = cache.find(x);
  if (i != cache.end()) return i->second;
  return cache[arg] = algoritmo_complejo(arg);
}

vector<int> genera_lista(vector<int> v) {
  vector<int> r;
  for (auto x : v) {
    r.push_back(calcula_clave(x));
  }
}
\end{lstlisting}
\end{block}
\begin{itemize}
  \item Se evita sincronización.
  \item Puede que se repita algún cálculo.
\end{itemize}
\end{frame}
