\subsection{Introducción}

\begin{frame}[fragile]{Exclusión mutua}
\begin{itemize}
  \item \cppid{mutex} permite controlar el acceso con exclusión mutua a un recurso.
    \begin{itemize}
      \item \cppid{lock()}: Adquiere el cerrojo asociado.
      \item \cppid{unlock()}: Libera el cerrojo asociado.
    \end{itemize}
\end{itemize}
\pause
\begin{block}{Ejemplo}
\begin{lstlisting}
mutex m;
int x = 0;

void f() {
  m.lock();
  ++x;
  m.unlock();
}

void g() {
  thread t1(f); thread t2(f);
  t1.join(); t2.join();

  cout << x << endl;
}
\end{lstlisting}
\end{block}
\end{frame}

\begin{frame}[fragile]{Problemas con \texttt{lock()}/\texttt{unlock()}}
\begin{itemize}
  \item Posibles problemas:
    \begin{itemize}
      \item Olvido de liberar cerrojo.
      \item Excepciones.
    \end{itemize}
\end{itemize}
\pause
\begin{block}{Solución unique\_lock}
\begin{lstlisting}
mutex m;
int x = 0;

void f() {
  unique_lock<mutex> l{m}; // Adquiere el cerrojo
  ++x;
} // Libera el cerrojo

void g() {
  thread t1(f); thread t2(f);
  t1.join(); t2.join();

  cout << x << endl;
}
\end{lstlisting}
\end{block}
\end{frame}

\begin{frame}[fragile]{Adquisición de múltiples \texttt{mutex}}
\begin{itemize}
  \item La función \cppid{lock()} permite adquirir a la vez varios \cppid{mutex}.
    \begin{itemize}
      \item Adquiere todos o ninguno.
      \item Si alguno está bloqueado espera dejando libres todos.
    \end{itemize}
\begin{lstlisting}
mutex m1, m2, m3;
void f() {
  lock(m1, m2, m3);
  // Acceso a datos compartidos

} // Cuidado: No se liberan los cerrojos
\end{lstlisting}
  \pause
  \item Especialmente útil en cooperación con \cppid{unique\_lock}
\begin{lstlisting}
void f() {
  unique_lock l1{m1, defer_lock}, unique_lock l2{m2, defer_lock};
  lock(l1, l2);
  // Acceso a datos compartidos

} // Ahora si se liberan los cerrojos
\end{lstlisting}
\end{itemize}
\end{frame}

\subsection{Mecanismos de espera}

\begin{frame}[fragile]{Esperas de tiempo}
\begin{itemize}
  \item Acceso al reloj.
\begin{lstlisting}
using namespace std::chrono;
auto t1 = high_resolution_clock::now();
\end{lstlisting}
  \item Diferencia de tiempos.
\begin{lstlisting}
auto dif = duration_cast<nanoseconds>(t2-t1);
cout << dif.count() << endl;
\end{lstlisting}
  \item Especificación de una espera.
\begin{lstlisting}
this_thread::sleep_for(microseconds{500});
this_thread::sleep_until(t + milliseconds{10});
\end{lstlisting}
\end{itemize}
\end{frame}

\begin{frame}[fragile]{Variables condición}
\begin{itemize}
  \item Mecanismo para sincronizar hilos en acceso a recursos compartidos.
    \begin{itemize}
      \item \cppid{wait()}: Espera en un \cppid{mutex}.
      \item \cppid{notify\_one()}: Despierta a un hilo en espera.
      \item \cppid{notify\_all()}: Despierta a todos los hilos en espera.
    \end{itemize}
\begin{lstlisting}
class peticion;

queue<peticion> cola;
condition_variable cv;
mutex m;

void productor();
void consumidor();
\end{lstlisting}
\end{itemize}
\end{frame}

\begin{frame}[fragile]{Productor}
\begin{lstlisting}
void consumidor() {
  for (;;) {
    unique_lock<mutex> l{m};

    while (cv.wait(l));

    auto m = cola.front();
    cola.pop();
    l.unlock();
   
    procesa(p);
  };
}
\end{lstlisting}
\begin{itemize}
  \item Efecto de \cppid{wait}:
    \begin{enumerate}
      \item Libera el cerrojo y espera una notificación.
      \item Adquiere el cerrojo al despertarse.
    \end{enumerate}
\end{itemize}
\end{frame}

\begin{frame}[fragile]{Consumidor}
\begin{lstlisting}
void productor() {
  for (;;) {
    peticion p = genera();

    unique_lock<mutex> l{m};
    m.push(p);

    cv.notify_one();
  }
}
\end{lstlisting}
\begin{itemize}
\item Efecto de \cppid{notify\_one()}:
  \begin{enumerate}
    \item Despierta a uno de los hilos esperando en la condición.
  \end{enumerate}
\end{itemize}
\end{frame}

