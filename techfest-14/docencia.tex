\section{Uso docente}

\begin{frame}[t]{Las JavaSchools}
  \begin{itemize}
    \item Una escuela Java se caracteriza por:
      \begin{itemize}
        \item Utiliza Java como primer lenguaje de programación.
        \item Ilustra en la mayoría de las asignaturas los conceptos mediante Java.
        \item Presta poca atención a paradigmas distintos de la orientación a objetos.
      \end{itemize}
    \pause
    \item Problema:
      \begin{itemize}
        \item La evolución de las tecnologías informáticas es demasiado rápida para hacer formación a corto plazo.
        \item La preparación universitaria debe centrarse en el medio y largo plazo.
      \end{itemize}
  \end{itemize}
\end{frame}

\begin{frame}[t]{Primer lenguaje}
  \begin{itemize}
    \item La gestión automática de memoria simplifica la programación.
      \begin{itemize}
        \item Pero los estudiante nos adquieren el patrón mental de adquisición/liberación.
        \item Es un patrón que va mucho más allá de la gestión de memoria.
      \end{itemize}
    \item \pause El modelo basado en máquina virtual oculta muchos detalles de la ejecución del programa.
      \begin{itemize}
        \item Esto favorece la confusión de conceptos que deben separarse claramente.
        \item Traducción, código fuente, código objeto, código ejecutable, compilación, enlace, ...
      \end{itemize}
    \item \pause El estudiante no adquiere preocupación por el rendimiento y la gestión responsable de recursos.
  \end{itemize}
\end{frame}

\begin{frame}[t]{(casi) Único lenguaje}
  \begin{itemize}
    \item A favor:
      \begin{itemize}
        \item Las asignaturas pueden centrarse en sus contenidos propios porque el alumno ya conoce el lenguaje.
        \item El estudiante acaba siendo un experto en el lenguaje.
      \end{itemize}
    \item \pause En contra:
      \begin{itemize}
        \item El estudiante no adquiere competencias para enfrentarse a situaciones nuevas o desconocidas.
        \item El estudiante intenta resolver todo tipo de problemas con una única herramienta.
        \item ¿ Y si no encuentra una solución razonable con el lenguaje?
        \item Factor limitante en el tipo de puestos a los que el titulado puedo optar.
      \end{itemize}
  \end{itemize}
\end{frame}

\begin{frame}[t]{Único paradigma}
  \begin{itemize}
    \item La Programación Orientada a Objetos es un paradigma de programación originado a finales de los 60.
      \begin{itemize}
        \item No es el único paradigma de programación.
        \item No se adapta bien a todos los problemas.
      \end{itemize}
  \end{itemize}
\end{frame}
