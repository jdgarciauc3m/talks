\section{Uso industrial}

\begin{frame}[t]{Popularidad}
\begin{center}
\begin{tabular}{|l|r|}
\hline
\textbf{Lenguaje} & \textbf{Cuota de popularidad}\\
\hline
C & 17.871\%\\
\hline
Java & 16.499\%\\	
\hline
Objective-C & 11.098\%\\
\hline
C++ & 7.548\%\\
\hline
C\# & 5.855\%\\	
\hline
PHP & 4.627\%\\
\hline
(Visual) Basic & 2.989\%\\
\hline
Python & 2.400\%\\
\hline
JavaScript & 1.569\%\\
\hline
Transact-SQL & 1.559\%\\
\hline
\end{tabular}
\vfill
{\tiny Fuente: TIOBE Index, enero de 2013}
\end{center}
\end{frame}

\begin{frame}[t]{Sistemas Operativos}
  \begin{itemize}
    \item Microsoft Windows: C, C++.
    \item Linux: C.
    \item Apple MacOS: C, C++, Objective-C.
    \item Sun Solaris: C.
    \item HP-UX: C.
    \item Apple iOS: C, C++, Objective-C.
    \item Google Android: C, Java.
    \item RIM Blackberry OS 4.x: C++.
    \item Amazon Kindle	OS: C.
  \end{itemize}
  \vfill
  {\tiny Fuente: http://www.lextrait.com/vincent/implementations.html}
\end{frame}

\begin{frame}[t]{Interfaz gráfica}
  \begin{itemize}
    \item Microsoft Windows UI: C++.
    \item Apple MacOS (Aqua): C++.
    \item Gnome: C.
    \item KDE: C++.
  \end{itemize}
  \vfill
  {\tiny Fuente: http://www.lextrait.com/vincent/implementations.html}
\end{frame}

\begin{frame}[t]{Ofimática}
  \begin{itemize}
    \item Microsoft Office: C++.
    \item Apache Open Office: C++.
    \item Corel Office: C++.
    \item Adobe Acrobat: C++.
    \item Evernote: C++.
  \end{itemize}
  \vfill
  {\tiny Fuente: http://www.lextrait.com/vincent/implementations.html}
\end{frame}

\begin{frame}[t]{Clientes de Internet}
  \begin{itemize}
    \item Navegadores Web:
      \begin{itemize}
        \item Microsoft Internet Explorer: C++.
        \item Mozilla Firefox: C++.
        \item Safari: C++.
        \item Google Chrome: C++.
        \item Opera: C++.
        \item Opera Mini: C++, Java, Pike.
        \item Mosaic: C.
      \end{itemize}
    \item Correo electrónico:
      \begin{itemize}
        \item Microsoft Outlook: C++.
        \item IBM Lotus Notes: C.
      \end{itemize}
    \end{itemize}
  \vfill
  {\tiny Fuente: http://www.lextrait.com/vincent/implementations.html}
\end{frame}

\begin{frame}[t]{Software de servidor}
  \begin{itemize}
    \item Bases de datos:
      \begin{itemize}
        \item Oracle: C++.
        \item MySQL: C++.
        \item IBM DB2: C.
        \item Microsoft SQL Server: C++.
        \item IBM Informix: C.
        \item SAP DB: C++.
      \end{itemize}
    \item Servidores Web:
      \begin{itemize}
        \item Apache: C, C++.
        \item Microsoft IIS: C++.
      \end{itemize}
  \end{itemize}
  \vfill
  {\tiny Fuente: http://www.lextrait.com/vincent/implementations.html}
\end{frame}

\begin{frame}[t]{Sitios Web}
  \begin{itemize}
    \item eBay: Java.
    \item PayPal: C++.
    \item Amazon: C++, Java.
    \item facebook: C++, PHP.
    \item Youtube: Python.
    \item Dropbox: Python.
  \end{itemize}
  \vfill
  {\tiny Fuente: http://www.lextrait.com/vincent/implementations.html}
\end{frame}

\begin{frame}[t]{Nichos de utilización}
  \begin{itemize}
    \item ¿Hay un lugar para lenguajes basados en máquina virtual?
      \begin{itemize}
        \item Aplicaciones ejecutadas en un servidor de aplicaciones.
        \item Aplicaciones sin requisitos estrictos de rendimiento.
        \item Aplicaciones sin restricciones de consumo de memoria.
      \end{itemize}
    \item ¿Tienen sentido los lenguajes basados en máquina virtual para aplicaciones de escritorio?
      \begin{itemize}
        \item La mayoría de los intentos han fracasado.
      \end{itemize}
    \item ¿Y para aplicaciones en dispositivos móviles?
      \begin{itemize}
        \item Google Android es un ejemplo.
        \item Pero si hace falta rendimiento o hay problemas de memoria se acaba recurriendo al NDK.
      \end{itemize}
  \end{itemize}
\end{frame}
